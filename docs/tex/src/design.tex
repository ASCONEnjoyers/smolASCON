\section{Design Rationale}
La suite è basata sullo \textbf{sponge design}. La funzione $p$ è basata su SPN (Substitution-Permutation-Network). Per questo le componenti principali del cifrario sono ispirate a schemi già standardizzati. 
\newline\newline
Il substitution layer utilizza una sostituzione affine equivalente alla S-Box utilizzata in \textsl{SHA-3}. Il permutation layer utilizza funzioni lineare simili a quelle utilizzate da SHA-2. 
\newline\newline
Lo sponge-based design in particolare si avvicina allo \textsl{SpongeWrap} e al \textsl{MonkeyDuplex}. Lo sponge-based design ha diversi vantaggi rispetto ad altre tecniche di costruzione come i tradizionali cifrari a blocchi o le modalità hash-based: 
\begin{itemize}
    \item Ampiamente studiate ad analizzate con diverse dimostrazioni sulla solo sicurezza. Utilizzate in SHA-3
    \item Design semplice, privo di key scheduling
    \item I blocchi di plaintext e ciphertext possono essere computati direttamente, senza attendere per la ricezione del messaggio completo o di conoscere la lunghezza di esso. 
    \item Utilizzo della medesima funzione sia per cifrare che per decifrare. 
\end{itemize} 
\subsection{Sponge Function}
TODO
\subsection{MonkeyDuplex}
TODO
\newpage